\documentclass{article}
\usepackage{amsmath,amsthm}
\usepackage{fullpage}

\begin{document}
\title{PHYS 32 Study Guide}
\author{Chris Barna (chris@unbrain.net)}
\date{April 2011}

\maketitle

\section{Simple Harmonic Motion}
\begin{equation}
  x=Acos(\omega t + \phi)
\end{equation}

Since our oscillating mass repeats its motion after a time equal to the period ($T$), and the sine or cosine function repeats itself every $2\pi$ radians, then:
\begin{equation}
  \omega = \frac{2 \pi}{T} = 2 \pi f = \sqrt{\frac{k}{m}}
\end{equation}

We call $\omega$ the \textbf{angular frequency}.
\begin{equation}
  T = 2 \pi \sqrt{\frac{m}{k}}
\end{equation}

\subsection{Energy in a Simple Harmonic Oscillator} % (fold)
\begin{equation}
  E = \frac{1}{2}kA^2 = \frac{1}{2}mv^2_{max}
\end{equation}

\begin{equation}
  v_{max} = \omega A
\end{equation}

\section{Pendulums} % (fold)
\label{sec:Pendulums}
The displacement of the pendulum along the arc is given by $x = l \theta$ where $\theta$ is the angle (in radians) that the cord makes with the vertical and $l$ is the length of the cord. The restoring force is equal to $F = -mg \sin \theta$.

\begin{equation}
  \omega = \sqrt{\frac{g}{l}}
\end{equation}
\begin{equation}
  \Theta = \Theta_{max} \cos (\omega t + \phi)
\end{equation}
So the frequency is: $f = \frac{1}{2\pi}\sqrt{\frac{g}{l}}$. The period is $T = 2\pi \sqrt{\frac{l}{g}}$.

\subsection{Physical Pendulum}
\begin{equation}
  T = 2 \pi \sqrt{\frac{I}{mgh}}
\end{equation}
% section Pendulums (end)

\section{Newton's Law of Universal Gravitation}
\begin{equation}
  F_{grav} = G \frac{m_1 m_2}{r^2}
\end{equation}

For sattelites that move in a circle:
\begin{equation}
  G \frac{m m_E}{r^2} = m \frac{v^2}{r}
\end{equation}

\subsection{Escape Velocity}
\begin{equation}
  v_{esc} = \sqrt{2GM_E/r_E}
\end{equation}


\subsection{Kepler's Laws of Planetary Motion}
\begin{enumerate}
  \item The path of each planet about the Sun is an elipse with the sun at one focus.
  \item Each planet moves so that an imaginary line drawn from the Sun to the planet sweeps out equal areas in equal periods of time (conservation of momentum).
  \item $\frac{T_1^2}{T_2^2} = \frac{s_1^3}{s_2^3}$; Generally $s_i = r_i$.
\end{enumerate}

\section{Fluids}
\begin{itemize}
  \item \textbf{Density ($\rho$):} $\rho = \frac{m}{V}$.
  \item \textbf{Pressure ($P$):} $P = \frac{F}{A}$. Units are \textbf{pascals} (Pa) which are equal to $1 N/m^2$. Pressure due to weight of a liquid is $P = \rho gh$ (potentially plus some initial pressure $P_0$.
  \item \textbf{Buoyancy ($F_b$):} $F_B = \rho_F V g$. At eqiulibrium, $\rho_{sub} V_{sub} g = \rho_F V_sub g$.
\end{itemize}

\subsection{Pascal's Principle}
\begin{eqnarray}
P_{out} &=& P_{in} \\
\frac{F_{out}}{A_{out}} &=& \frac{F_{in}}{A_{in}}
\end{eqnarray}

\subsection{Fluids in Motion}
\begin{equation}
  \rho_1 A_1 v_1 = \rho_2 A_2 v_2
\end{equation}
Since ideal fluids are incompressible, $\rho_1 = \rho_2$.

\subsubsection{Bernoulli's equation}
\begin{equation}
  P_1 + \frac{1}{2}\rho v_1^2 + \rho g y_1 = P_2 + \frac{1}{2} \rho v_2^2 + \rho g y_2
\end{equation}

\section{Waves}
\begin{equation}
  y(x, t) = A \sin (kx - \omega t)
\end{equation}
\begin{itemize}
  \item \textbf{Wavelength ($\lambda$)}
  \item \textbf{Angular Wave Number ($k$):} $\frac{2 \pi}{\lambda}$
  \item \textbf{Wave Number ($K$):} $\frac{1}{\lambda}$
  \item \textbf{$\omega$:} $\frac{2 \pi}{T}$
  \item \textbf{Wave Speed ($v$):} $v = \frac{\omega}{k} = \frac{2 \pi}{T} \frac{\lambda}{2 \pi} = \frac{\lambda}{T} = \lambda f = \sqrt{\frac{T}{\mu}}$
  \item $\mu = \frac{m}{l}$
  \item \textbf{Intensity ($I$):} $\frac{\bar{P}}{S}$ where $S$ typically equals $4 \pi r^2$ and $\bar{P} = \frac{E}{t}$ 
\end{itemize}

\subsection{Standing Wave} % (fold)
\label{sub:StandingWave}
\begin{equation}
  y = y_1 + y_2 = 2A\sin (kx) \cos (\omega t)
\end{equation}

\subsection{Decibels}
\begin{equation}
  \beta = 10 \log \frac{I}{I_0}
\end{equation}

\subsection{Fundamental Frequencies} % (fold)
\begin{itemize}
  \item \textbf{Attached At Both Ends:} $l = \frac{n}{2} \lambda_1$ where $n = 1, 2, 3,\dots$
  \item \textbf{One End Closed:} $l = \frac{n}{4} \lambda_1$ where $n = 1, 3, 5,\dots$
  \item \textbf{Open At Both Ends:} $l = \frac{n}{2} \lambda_1$ where $n = 1, 2, 3,\dots$
\end{itemize}

\subsection{Beat Frequency}
\begin{equation}
  f_{beat} = f_2 - f_1
\end{equation}

\subsection{Doppler Effect}
\begin{itemize}
  \item \textbf{Moving Source:} $f' = f(\frac{v_{snd}}{v_{snd} \pm v_s})$ ($-$ is towards, $+$ is away)
  \item \textbf{Moving Detector:} $f' = f(\frac{v_{snd} \pm v_d}{v_{snd}})$ ($+$ is towards, $-$ is away)
  \item \textbf{Both Are Moving:} $f' = f(\frac{v \pm v_d}{v \pm v_s})$
\end{itemize}

\section{Light} % (fold)
\label{sec:Light}
\begin{itemize}
  \item \textbf{Snell's law of refraction:} $n_1 \sin \theta_1 = n_2 \sin \theta_2$
\end{itemize}

When a light wave travels from one medium to another, its frequency does not change but it's wavelength does.

\subsection{Thin Film Interference} % (fold)
If a substance is sitting on another substance and the higher substance has a lower index of refraction. Part of light of a uniform wavelength is refracted off each layer. If the difference in path between the light off the top and bottom layers is a whole number of wavelengths, then constructive interference occurs. Similarly, if it is $\frac{1}{2}, \frac{3}{2} , \dots$ there will be destructive interference.

However, if "a beam of light reflected by a material with index of refraction greater than that of the material in which it is traveling, changes phase by 1/2 cycle.
% subsection Thin Film Interference (end)
% section Light (end)

\section{Thermodynamics} % (fold)
\label{sec:Thermodynamics}
\begin{itemize}
  \item \textbf{Reservoir:} Has as much heat as you want, will hold as much heat as you want.
  \item \textbf{Zeroth Law:} If bodies $a$ and $b$ are each in thermal equilibrium with body $c$ then they are in thermal equilibrium with each other.
  \item \textbf{Linear Thermal Expansion:} $\Delta l = \alpha l_0 \Delta T$. $\alpha$ is the coefficient of linear expansion.
  \item \textbf{Volumetric Thermal Expansion:} $\Delta V = \beta V_0 \Delta T$
  \item \textbf{Boyle's Law:} $PV = c$
  \item \textbf{Charles's Law:} The volume of a given amount of gas is directly proportional to the absolute temperature when the pressure is kept constant.
  \item \textbf{Gay-Lussac's law:} At constant volume, the absolute pressure of a gas is directly proportional to the absolute temperature.
  \item \textbf{Ideal Gas Law:} $PV = nRT$ where $n$ is the number of mols you have (mass / molecular mass) and $R = 8.314J$
  \item \textbf{Heat ($Q$):} $Q = mc \Delta T$. Heat lost is equal to heat gained.
  \item \textbf{First Law of Thermodynamics:} $\Delta E_{int} = Q - W$ or ($\Delta K + \Delta U + \Delta E_{int} = Q - W$). $\Delta E_{int} = \frac{3}{2}nR \Delta T$.
  \item \textbf{Isothermal Processes ($\Delta T = 0$):} Ideal gas law stands. $W = Q = nRT ln \frac{V_B}{V_A}$.
  \item \textbf{Adiabatic Processes ($Q=0$):} $\Delta E_{int} = -W$. $PV^\gamma = c$. $\gamma = \frac{C_P}{C_V}$
  \item \textbf{Isobaric Processes ($P=c$):} $Q = \Delta E_{int} + W$. $W = nRT_B (1 - \frac{V_A}{V_B})$.
  \item \textbf{Isovolumetric Processes ($V=c$):} $W=0$ so $Q = \Delta E_{int}$
  \item \textbf{Second Law of Thermodynamics:} Heat can flow spontaneously from a hot object to a cold object; heat will not flow spontaneously from a cold object to a hot object.
  \item \textbf{Heat Engines:} $Q_H = W + Q_L$ by conservation of energy. Efficiency is defined as $e = \frac{W}{Q_H} = \frac{Q_H - Q_L}{Q_H}$.
  \item \textbf{Carnot efficiency:} A reversible engine. $e_{ideal} = 1 - \frac{T_L}{T_H}$
  \item \textbf{Entropy:} $\Delta S = \frac{Q}{T}$. $\Delta S = S_b - S_a = \int_a^b dS = \int_a^b \frac{dQ}{T}$.
\end{itemize}

% section Thermodynamics (end)
\end{document}
