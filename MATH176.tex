\documentclass{article}
\usepackage{amsmath,amsthm}
\usepackage{fullpage}

\begin{document}

\title{MATH 176 (Combinatorics) Midterm Review}
\author{Chris Barna (chris@unbrain.net)}
\date{April 2011}
\maketitle

\section{Counting Principles}

\subsection{The Addition Principle}
If there are $r_1$ different objects in the first set, $r_2$ different
objects in the second set, \dots, and $r_m$ different objects in the
$m$th set, and \emph{if the different sets are disjoint}, then the
number of ways to select an object from one of the $m$ sets is
$r_1+r_2+\cdots+r_m$.

\subsection{The Multiplication Principle}
Suppose a procedure can be broken into $m$ successive (ordered) stages,
with $r_1$ different outcomes in the first stage, $r_2$ different
outcomes in the second stage, \dots, and $r_m$ different outcomes in the
$m$th stage. If the number of outcomes at each stage is independent of
the choices in previous stages and \emph{if the composite outcomes are
all distinct}, then the total procedure has $r_1\times r_2 \times \cdots
\times r_m$ different composite outcomes.

\section{Simple Arrangements and Selections}
\subsection{Permutations: $P(n,r)$}
A \textbf{permutation} of $n$ distinct objects is an arrangement, or
ordering, of the $n$ objects. An \textbf{$r$-permutation} of $n$
distinct objects is an arrangement using $r$ of the $n$ objects.
\begin{center}
$P(n,k) = \frac{n!}{(n-k)!}$
\end{center}

\subsubsection{Arrangements with Repetition}
If there are $n$ objects with $r_1$ of type 1, $r_2$ of type 2, \dots,
and $r_m$ of type $m$, where $r_1 + r_2 + \cdots + r_m = n$, then the
number of arrangements of these $n$ objects, denoted $P(n; r_1,
r_2,\dots,r_m)$ is:

\begin{center}
$P(n;r_1,r_2,\dots,r_m)={n \choose r_1}{n-r_1 \choose r_2}{n-r_1-r_2
\choose r_3}\cdots{n-r_1-r_2-\cdots-r_{m-1} \choose r_m} = \frac{n!}{r_1!r_2!\dots r_m!}$
\end{center}

\subsection{Combinations: $C(n,r)$}
An \textbf{$r$-combination} of $n$ distinct objects is an unordered
selection, or \emph{subset}, of $r$ out of the $n$ objects.

\begin{center}
$C(n,r)=\frac{P(n,r)}{P(r,r)}=\frac{n!}{r!(n-r)!}$
\end{center}

The numbers $C(n,r)$ are frequently called \textbf{binomial
coefficients} because of their role in the binomial expansion $(x+y)^n$.

\subsubsection{Selections with Repetition}
The number of selections with repetition of $r$ objects chosen from $n$
types of objects is $C(r+n-1,r)$.

\section{Distributions}
\subsection{Distinct Objects}
The process of distributing $r$ distinct objects into $n$ different
boxes is equivalent to putting the distinct objects in a row and
stamping one of the $n$ different box names on each object. The
resulting sequence of box names is an arrangement of length $r$ formed
from $n$ items (box names) with repetition. Thus there are $n \times n
\times \cdots \times n (r\thinspace ns) = n^r$ distributions of the $r$ distinct
ojbects. If $r_i$ objects must go in box $i$, $1 \le i \le n$, then
there are $P(r; r_1, r_2,\dots,r_n)$ distributions.

\subsection{Identical Objects}
The process of distributing $r$ identical objects into $n$ different
boxes is equivalent to choosing an (unordered) subset of $r$ box names
with repetition from among the $n$ choices of boxes. Thus there are
$C(r+n-1,r) = \frac{(r+n-1)!}{r!(n-1)!}$ distributions of $r$ identical
objects.

\section{Binomial Identities}
\subsection{Binomial Theorem}
$(1-x)^n={n \choose 0}+ {n \choose 1}x+{n \choose 2}x^2+\cdots+{n
\choose k}x^k + \cdots + {n \choose n}x^n$

\subsection{Identities}
\begin{enumerate}
\item ${n \choose 0} + {n \choose 1} + {n \choose 2} + \cdots + {n
\choose n} = 2^n$
\item ${n \choose 0} + {n+1 \choose 1} + {n+2 \choose 2} + \cdots + {n+r
\choose r} = {n+r+1 \choose r}$
\item ${r \choose r} + {r+1 \choose r} + {r+2 \choose r} + \cdots + {n
\choose r} = {n+1 \choose r+1}$
\item ${n \choose 0}^2 + {n \choose 1}^2 + {n \choose 2}^2 + \cdots + {n
\choose n}^2 = {2n \choose n}$
\end{enumerate}
\end{document}
