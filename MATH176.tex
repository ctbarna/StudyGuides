\documentclass{article}
\usepackage{amsmath,amsthm}
\usepackage{fullpage}

\begin{document}

\title{MATH 176 (Combinatorics) Midterm Review}
\author{Chris Barna (chris@unbrain.net)}
\date{April 2011}
\maketitle

\section{Counting Principles}

\subsection{The Addition Principle}
If there are $r_1$ different objects in the first set, $r_2$ different
objects in the second set, \dots, and $r_m$ different objects in the
$m$th set, and \emph{if the different sets are disjoint}, then the
number of ways to select an object from one of the $m$ sets is
$r_1+r_2+\cdots+r_m$.

\subsection{The Multiplication Principle}
Suppose a procedure can be broken into $m$ successive (ordered) stages,
with $r_1$ different outcomes in the first stage, $r_2$ different
outcomes in the second stage, \dots, and $r_m$ different outcomes in the
$m$th stage. If the number of outcomes at each stage is independent of
the choices in previous stages and \emph{if the composite outcomes are
all distinct}, then the total procedure has $r_1\times r_2 \times \cdots
\times r_m$ different composite outcomes.
\end{document}
