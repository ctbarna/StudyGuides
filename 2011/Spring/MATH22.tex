\documentclass{article}
\usepackage{amsmath,amsthm}
\usepackage{fullpage}

\begin{document}
\title{MATH 22 (Differential Equations) Study Guide}
\author{Chris Barna (chris@unbrain.net)}
\date{April 2011}

\maketitle

\section{Definitions}
\begin{enumerate}
  \item \textbf{Differential equation:} any equation that has a derivative. Ex: $\frac{d^2 y}{dx^2} + xy = \frac{dy}{dx}$.
  \item \textbf{Partial differential equation:} A differential equation that has a partial derivative.
  \item \textbf{Ordinary differential equation:} A differential equation that does not have partial derivaties.
  \item \textbf{Order:} The order of the highest ordered derivative in a differential equation.
  \item \textbf{Degree:} The algebraic degree of the highest ordered derivative.
  \item \textbf{Linear differential equation:} A differential equation that is linear in the dependent variable and its derivatives.
\end{enumerate}

\section{Separation of Variables}
\begin{enumerate}
  \item \textbf{Step 1.} Separate variables.
  \item \textbf{Step 2.} ...
  \item \textbf{Step 3.} Profit.
\end{enumerate}

\section{Homogenous Functions}
Function $f(x,y)$ is homogenous of degree $k$ if $f(\lambda x, \lambda y) = \lambda^k f(x,y)$. If coefficients of $dx$ and $dy$ are homogenous of the same degree, use substitution. Let $x=vy$ or $y=vx$.

\section{Exact Differential Equations}
An equation is an exact differential equation if there exists a function $F(x,y)=c$ such that the total differential of $F(x,y)=Mdx+Ndy$.
\begin{eqnarray}
Mdx + Ndy = 0 \\
M = \frac{\partial F}{\partial x}, \: N = \frac{\partial F}{\partial y} \\ 
My = \frac{\partial^2 F}{\partial x \partial y} = Nx \\
My = Nx
\end{eqnarray}

\section{Linear, First Order Differential Equations}
\begin{eqnarray}
a(x)\frac{dy}{dx} + b(x)y &=& c(x) \\
\frac{dy}{dx} + \frac{b(x)}{a(x}y &=& \frac{c(x)}{a(x)} \\
\frac{dy}{dx} + P(x)y &=& Q(y) \\
dy + (Py - Q)dx &=& 0\\
v(x)dy + [Pv(x)y - v(x)Q]dx &=& 0 \\
v(x) = e^{\int{Pdx}}
\end{eqnarray}

Then solve the equation as an exact differential equation.

\section{Integrating factors}

Suppose that a function $u$ is to be an integrating factor of $Mdx+Ndy=0$. Then the equation $uMdx+uNdy=0$ must be exact. Therefore,
\begin{eqnarray}
\frac{\partial}{\partial y}(uM) &=& \frac{\partial}{\partial x}(uN) \\
u \frac{\partial M}{\partial y} + M \frac{\partial u}{\partial y} &=& u \frac{\partial N}{\partial x} + N \frac{\partial u}{\partial x} \\
u ( \frac{\partial M}{\partial y} - \frac{\partial N}{\partial x} ) &=& N \frac{\partial u}{\partial x} - M \frac{\partial u}{\partial y}
\end{eqnarray}

If $u$ is a function of $x$ alone, then $\partial u / \partial y = 0$ and $\partial u / \partial x$ becomes $du/dx$. Leading to:
\begin{eqnarray}
\frac{1}{N}(\frac{\partial M}{\partial y} - \frac{\partial N}{\partial x})dx=\frac{du}{u} = f(x)
\end{eqnarray}
So the integracting factor is $u = e^{\int f(x) dx}$. Similarly:
\begin{eqnarray}
\frac{1}{M}(\frac{\partial M}{\partial y} - \frac{\partial N}{\partial x})=-g(y)
\end{eqnarray}

\section{Linear Coefficients in Two Variables}
\begin{eqnarray}
(a_1 x + b_1 y + c_1)dx + (a_2 x + b_2 y + c_2)dy = 0 \\
a_1 x + b_1 y + c_1 = 0 \; a_2 x + b_2 y + c_2 = 0
\end{eqnarray}

\subsection{Intersection}
Suppose they intersect at $(h,k)$. Let $x=u+h,\: y=v+k$. You end up with the homogenous differential equation:
\begin{eqnarray}
(a_1 u + b_1 v)du + (a_2 u + b_2 v)dv = 0
\end{eqnarray}

\subsection{Parallel}
Suppose they are parallel, then $\frac{a_1}{b_1}=\frac{a_2}{b_2}$ so $a_2 = k a_1$. Same for $b$. Then:
\begin{eqnarray}
(a_1 x + b_1 y + c)dx &+& (k a_1 x + k b_1 y + c_2)dy = 0 \\
&+&[k(a_1 x + b_1 y) + c_2]dy = 0
\end{eqnarray}

Let $w=a_1 x + b_1 y$ and $dw = a_1 dx + b_1 dy$. Then separate variables.
\end{document}
