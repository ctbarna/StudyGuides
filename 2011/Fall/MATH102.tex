\documentclass{article}
\usepackage{amsmath,amsthm}
% \usepackage{fullpage}

\begin{document}
\title{Math 102 (Advanced Calculus) Study Guide}
\author{Chris Barna (chris@unbrain.net)}
\date{Fall 2011}

\maketitle

\section{Functions, Sets, etc.}
\begin{enumerate}
  \item A \textbf{function} $y=f(x)$ is defined by a relation between $x$ and
    $y$ where there is assigned one and only one value for $y$ for each $x$.
  \item A \textbf{neighborhood} of radius $r$ of a point $x$,
    $S_r: {y|d(x,y) \subset r}$.
  \item Let $X$ be the univers. A set $A \subset X$ is an \textbf{open set} if,
    for every point $x \in A$ there exists at least one neighborhood $S_r(x)$
    such that $S_r(x) \subset A$.
  \item A set $A$ is \textbf{closed} if the compliment of $A$ ($A'$) is closed.
  \item A point $x$ is a \textbf{limit point} of $G$ if every neighorhood of
    $x$ contains at least one point of $G$ other than $x$.
  \item A point $x$ is a \textbf{boundary point} if every neighborhood of $x$
    contains at least one point in $G$ and one point in $G'$.
  \item A set $G$ is bounded if it can be enclosed by a neighborhood of finite
    radius.
  \item A domain is a set where any two points can be connected by a broken
    line.
\end{enumerate}

\section{Limits}
\begin{enumerate}
  \item A function $f(x)$ has a \textbf{limit} $L$ as $x$ approaches $b$ if,
    given any $\epsilon > 0$, there exists a $\delta > 0$ such that $|f(x) - L| <
    \epsilon$ whenever $|x-b| < \delta$ except possibly at $x=b$.
  \item A function $f(x,y)$ has a limit $L$ as $x \to a$, $y \to b$ if, given
    any $\epsilon > 0$, there exists a $\delta > 0$ such that
    $|f(x,y)-L|<\epsilon$ wherever $(x-a)^2+(y-b)^2 < \delta^2$.
\end{enumerate}

\subsection{Properties}
If $\lim_{x \to a, y \to b} f(x,y)= L_1$ and
$\lim_{x \to a, y \to b} g(x,y) = L_2$, then:
\begin{enumerate}
  \item $\lim[f(x,y)+g(x,y)] = L_1 + L_2$
  \item $\lim kf(x,y) = kL_1$
  \item $\lim f(x,y)g(x,y) = L_1L_2$
  \item $\lim \frac{f(x,y)}{g(x,y)} = \frac{L_1}{L_2}$
  \item If $\lim_{x \to a^+} f(x) = lim_{x \to a^-} f(x)$ then the limit
    exists. For multivariable functions, you need to find a problem path to
    prove a limit doesn't exist. No easy way to prove a limit exists.
\end{enumerate}

\section{Partial Derivatives}
\begin{equation}
  f_x(a,b) = \lim_{\Delta x \to 0} \frac{f(a+\Delta x, b) - f(a,b)}{\Delta x} =
  \frac{\partial z}{\partial x}_{(a,b)}
\end{equation}

\subsection{$\Delta z$}
Let $z=f(x,y)$ be continuous in a domain which posesses partial derivatives
$f_x$ and $f_y$. Let $f_x$ and $f_y$ be continuous at $(a,b)$. Let
$\Delta z = f(a+\Delta x, b+ \Delta y) - f(a,b)$. Then:
\begin{equation}
  \Delta z = f_x(a,b) \Delta x + f_y(a,b)\Delta y +
  \epsilon_1 \Delta x \epsilon_2 \Delta y
\end{equation}

Where $\epsilon_1, \epsilon_2 \to 0$ as $\Delta x, \Delta y \to 0$.

\subsection{Chain rule}
If $z=f(x,y)$, $x = g(u,v,w)$, and $y=h(u,v,w)$, then:

\begin{equation}
  \frac{\partial z}{\partial w} = \frac{\partial f}{\partial x}
  \frac{\partial x}{\partial w} + \frac{\partial f}{\partial y}
  \frac{\partial y}{\partial w}
\end{equation}

\end{document}
