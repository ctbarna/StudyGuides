\documentclass{article}
\usepackage{amsmath,amsthm,amsfonts}
\usepackage{fullpage}
\setlength{\parindent}{0pt}
\setlength{\parskip}{1ex}

\begin{document}
\title{MATH 178 (Cryptography) Study Guide}
\author{Chris Barna (chris@unbrain.net)}
\date{Fall 2011}

\maketitle

\section{Types of Ciphers/Systems}
\begin{enumerate}
  \item \textbf{Stream Ciper}: Operates on a message symbol by symbol or, now,
    bit by bit.
  \item \textbf{Block Cipher}: Operates on blocks of symbols or bits.
  \item \textbf{Transposition Cipher}: Rearranges letters in a plaintext.
  \item \textbf{Substitution Cipher}: Replaces letters, symbols, etc. with
    others but doesn't change the order.
  \item \textbf{Product Cipher}: Alternates substitution and transposition.
  \item \textbf{Symmetric Key Cryptosystem}: Requires a secret shared key. Key
    must be agreed upon ahead of time.
  \item \textbf{Public Key Cryptosystem}: Each user has an encrypting key which
    is published and a decrypting key which is not.
\end{enumerate}

\section{Historical Ciphers}

\subsection{Spartan Scytale}
Example of a transposition cipher. Letters were written on a long strip of
leather that was wrapped around a cylinder. The diameter of the cylinder was
the key to decreption.

\subsection{Playfair}
Key is a word. Write the word, without repeating letters, as the first letters
of a 5x5 square. Write the remaining letters of the alphabet in order to fill
in the square. I and J occupy the same space.

\textbf{To encrypt}: Working two letters at a time, make a box with a pair of
letters and take the letters from the opposite corners of the box. If the
letters are in the same row, replace each letter with the letter to the right.
If the letters are in the same column, replace each letter by the letter below
it.

\subsection{ADFGVX}
Product cipher with fixed table.

\textbf{To encrypt}: Replace a PT letter/digit by the (row, column) pair.
Choose a key with no repeated letters (ex: DEUTSCH). Number the letters in the
key alphabetically. Put partial ciphertext row by row under the key. Write the
columns numerically in increasing order.

\subsection{Shift Enciphering Transformation}
If we have $N$ letters, shift enciphering $C = P + b (\text{mod} N)$ where $b$
is the encrypting key. Decrypting key is $-b$.

\subsection{Affine Cipher}
$C = aP + b (\text{mod} N$ where $a$, $b$ is the encrypting key.
$\text{gcd}(a,n) = 1$ otherwise it doesn't work. Total number of key pairs is
thus $\phi(N) \times (N)$.

\subsection{Digraphs}
Compute digraphs with $C = a(26) + b$.

\subsection{Stream Cipher}
Plaintext is encoded in ASCII and then this is XORed with a binary keystream.

\subsection{Pseudo-Random Bit String Generation}
Say $p$ is a large prime for which $2$ generates $\mathbb{F}_p^*$ and assume
$q=2p+1$ is also prime. Let $g$ be a specified generater of $\mathbb{F}_q^*$.

The key $k$ is a number with $\text{gcd}(k,2p)=1$. Let $s_1=g^k$ in
$\mathbb{F}_q^*$ and $s_1 (\text{mod} 2) = k_1$. For $i \geq 1$ let
$S_{i+1}=S_i^2 \in \mathbb{F}_q^*$.

\section{Number Theory}

\subsection{Modular Arithmetic}
In general, working mod $m$ breaks integers into $m$ subsets. Each subset
contains exactly one representative in the interval $0,...,m-1$.
$\mathbb{Z}/m\mathbb{Z}$ has $m$ elements.

\subsubsection{Solving $ax \equiv b (\text{mod} m)$}
\begin{enumerate}
  \item If $\text{gcd}(a,m) = 1$ then $a^{-1}$ exists so
    $x\equiv a^-1 \cdot b$.
  \item If $\text{gcd}(a,m) = g > 1$ and if $g | b$ then same as
    $\frac{a}{g}x\equiv\frac{b}{g} (\text{mod} \frac{m}{g})$
  \item If $\text{gcd}(a,m) = g > 1$ and if $g \not| b$ then no solution.
\end{enumerate}

\subsubsection{Properties}
\begin{enumerate}
  \item $a \equiv a (\text{mod} m)$.
  \item if $a \equiv b (\text{mod} m)$ then $b \equiv a (\text{mod} m)$.
  \item if $a \equiv b (\text{mod} m)$ and $b \equiv c (\text{mod} m)$, then
    $a \equiv c (\text{mod} m)$.
  \item if $a \equiv b (\text{mod} m)$ and $c \equiv d (\text{mod} m)$, then
    $a \pm c \equiv b \pm d (\text{mod} m)$.
\end{enumerate}

\subsection{$\phi(n)$}
Counts number of coprimes less than $n$.

\begin{enumerate}
  \item $\phi(p) = p-1$
  \item $\phi(p^r) = p^{r-1}(p-1)$
  \item If $\text{gcd}(m,n) = 1$ then $\phi(mn)= \phi(m) \phi(n)$
\end{enumerate}

\subsection{$\mathbb{Z} / n\mathbb{Z}$}
An element $x$ of $\mathbb{Z}/m\mathbb{Z}$ has a multiplicative inverse denoted
$x^{-1}$. in $\mathbb{Z}/m\mathbb{Z}$ if $\text{gcd}(x,m)=1$. The elements of
$\mathbb{Z}/m\mathbb{Z}$ with inverses are denoted $\mathbb{Z}/m\mathbb{Z}^*$.

Operations in $\mathbb{Z}/m\mathbb{Z}$ are $+, -, \cdot$. Operations in
$\mathbb{Z}/m\mathbb{Z}^*$ are $\cdot, \div$.

Size of $\mathbb{Z}/n\mathbb{Z}^*$ is $\phi(n)$.

\subsection{Finite Fields}
If $p$ is prime then $\mathbb{Z}/p\mathbb{Z}$ is denoted $\mathbb{F}_p$. All
non-zero elements, $\alpha$, have $\text{gcd}(\alpha, p) = 1$.

\subsubsection{Finite Fields over Polynomials}
Let $\mathbb{F}_2[x]$ be the set of polynomials with coefficients in
$\mathbb{F}_2$. If you take $\mathbb{F}_2[x]$ and reduce it by an irreducable
polynomial of degree $d$, you get all the polynomials of degree lower than $d$.
This is denoted $\mathbb{F}_{2^d}$. $\mathbb{F}_{2^d}^{*}$ is the non-zero
elements. Easy to represent these polynomials in a computer.

\subsubsection{Finite Field Discrete Log Problem}
Let $\mathbb{F}_q$ be a finite field. Let $g \in \mathbb{F}_q^*$ be a
generator. Let $b \in \mathbb{F}_q^*$. Then $g^i = b$ for som positive integer
$i \leq q-1$. Determining $i$ given $\mathbb{F}_q^*$, $g$ and $b$ is the FFDLP.

\subsection{Repeated Squares Algorithm}
To reduce $b^n(\text{mod}m)$, first write $n$ in base 2
$(v[k],v[k-1],...,v[1],v[0])_2$. Let $a$ denote a partial product. At the
start, $a=0$.

If $v[0]=1$, change $a$ to $b$.

\begin{enumerate}
  \item Reduce $b^2(\text{mod}m)=b[1]$. If $v[1]=1$ replace $a$ by reduction of
    $a * b[1](\text{mod}m)$, else no change in $a$.
  \item Repeat \#1 for each digit in $v$. $b^n(\text{mod}m)=a$.
\end{enumerate}

\subsection{Elliptic Curves}
An elliptic curve is a curve described by an equation of the form
$y^2+a_1xy+a_3y=x^3+a_2x^2+a_4x+a_6$ and has an extra point called the zero
point. 0 point is a point at $\infty$. 3 points on a line should sum to the 0
point.

\begin{enumerate}
  \item $P+Q+0=0$ so $Q=-P$. Two points, not 0, with the same x-coord are
    negatives of each other.
  \item $P+Q=-R$. If you draw a line through $P$ and $Q$, you can find $R$,
    then negate it.
  \item $2T=-U$. Neede to implicitely differentiate elliptic curve at $P$.
\end{enumerate}

We can also use elliptic curves with finite fields. $E(\mathbb{F}_n)$ is the
set of points on $E$ with coordinates in $\mathbb{F}_n$. Some elliptic curves
need 2 generators.

\subsubsection{Elliptic Curve Discrete Log Problem}
The set $E_2(\mathbb{F}_{109})$ is generated by $(0,1)$. The point $(39,45)$ is
in this set. So $(39,45)=(0,1)+(0,1)+...+(0,1)=n(0,1)$. Finding $n$ is the
ECDLP.

\section{S-AES}
In 1997, NIST held a competition to replace DES. In 2001, NIST chose 128-bit
Rijndael with 128 bit key to become the Advanced Encryption Standard.

AES has full diffusion after two rounds and is non-linear allowing for
reasonable security. AES is also fairly efficient. Each step can be broken into
independent calculations.

Simplified AES (Schaefer, Musa, Wedig) operates over the finite fielt
$\mathbb{F}_16 = \mathbb{F}_2[2]/(x^4+x+1)$.

\subsection{Operations}
Each function operates on a state, consisting of 4 nibbles.

\subsubsection{S-box}
Maps from nibbles to nibbles. Can be inverted. Look it up.

\subsubsection{Key Expansion}
Original key fills $W[0]$ and $W[1]$. For $W[i]$ where $2 \leq i \leq 5$:

\begin{enumerate}
  \item if $i \equiv 0 (\text{mod} 2)$ then
    $W[i]=W[i-2]\oplus\text{RCON}(i/2)\oplus\text{SubNib}(\text{RotNib}(W[i-1]))$
  \item if $i \not\equiv 0 (\text{mod} 2)$ then $W[i]=W[i-2]\oplus W[i-1]$
\end{enumerate}

$\text{RotNib}(N_0N_1) = N_1N_0$.
$\text{SubNib}(N_0N_1) = \text{S-box}(N_0)\text{S-box}(N_1)$.
$\text{RCON}[1]=10000000$, $\text{RCON}[2]=00110000$.

\subsubsection{$A_{K_i}$}
Add key. XOR $K_i$ with the state.

\subsubsection{$NS$}
Nibble substitution. Replace each nibble $N_i$ in a state by
$\text{S-box}(N_i)$. Do not change the order of the nibbles.

\subsubsection{$SR$}
Shift row. Shift the bottom row of the state.

\subsubsection{$MC$}
Mix column. Look it up.

\section{Public Key Cryptography}
In public key cryptography, every user has a public key and a private key.
There is no known way of quickly getting the private key from the public key.

\subsection{RSA}
Pick two different primes, $p$, $q$ both around $10^{150}$. Compute $n=pq$ and
$\phi(n)=(p-1)(q-1)$. Pick a number $e$ where $\text{gcd}(e,\phi(n)=1$ and
compute $e^{-1}(\text{mod}\phi(n))=d$. Publish $n$, $e$.

To \textbf{encrypt}, take a message, encode it, raise the integer to the
receiver's $e$ mod the receiver's $n$.

To \textbf{decrypt}, take the ciphertext, raise it to your $d$ mod your $n$ and
decode.

To \textbf{sign}, raise a message to your $d$ and send it. The receiver will
know that the signature is correct when they raise it to your $e$ mod your $n$.

There are different methods to sign:
\begin{enumerate}
  \item Sender could append some standard signature at the end of every
    message, $M^d(\text{mod}n)$. This is easy to steal because it never
    changes.
  \item Sender could encrypt a message using AES, send that. Take the original
    message, hash it and sign it ($X$). Receiver decrypts the AES, hashes it.
    Receiver raises the signature to $e$ and compares the two hashes.
  \item What if they don't want Eve reading the hash? Sender can encrypt $X$
    with RSA. If $n_S < n_{R}$, nothing will go wrong, Sender can compute
    $(X^{d_S} (\text{mod}n_S))^{e_R}(\text{mod}n_R)$. If $n_S > n_R$, encrypt
    first and then sign.
\end{enumerate}

\subsection{Diffie-Hellman Key Agreement}
\subsubsection{Finite Field}
For a set of users, fix $\mathbb{F}_q$ and $g$. Each user has a private key
number $a$ where $1 < a < q-1$ and a public key, the reduction of $g^a$ in
$\mathbb{F}_q$. Each user publishes their $g^a$.

Alice and Bob each compute $g^{a_Aa_b}$ at the beginning to agree on a key.

\subsubsection{Elliptic Curve}
Choose a finite field $\mathbb{F}_q$ with $q=2^{163}$. Fix some elliptic curve
$E$ whose defining equation has coefficients in $\mathbb{F}_q$ and a
(pseudo)generating point $G=(x,y) \in E(\mathbb{F}_q)$. The point $G$ should
have the property that the first positive integer $n$ with $nG=0$ should be
very large.

Each user has a private key, number $n$ where $0 << a << n$. $aG$ are the
public keys. Alice and Bob use $a_Aa_BG$ as their shared key.

\subsection{ElGamal Message Exchange}
\subsubsection{Finite Field}
Everybody uses the same $\mathbb{F}_q^*$ and $g$. A will send a message $M$ to
Bob. Alice chooses a random $k$, $1<k<q$ (session key).She then sends $g^k$ and
$Mg^{a_Bk}$ reduced in $\mathbb{F}_q$.

Bob computes $(g^k)^a_B$ and then computes $Mg^{a_Bk}(g^{a_Bk})^{-1}$

\subsubsection{Elliptic Curve}
The issue with encoding a message on a point is that the point may not exist in
the given finite field. If we work with $\mathbb{F}_{29}$, we can encode the
alphabet for $y^2=x^3-4$. If we try to encode, $l=11$ as the x-coordinate but
their may be no point with $x=11$ on $E(\mathbb{F}_{29})$.

Instead, we could use $\mathbb{F}_{257}$. Encode a message and have one free
digit as x-coord of a point.

Alice encodes a message and gets a point $Q$. Alice then looks up Bob's public
key $a_BG$. Alice also chooses a random session key $k$. Alice finds $Q+ka_BG$
and $kG$ (points) and sends them to Bob.

To decrypt, Bob finds $a_B(kG)$ then subtracts: $(Q+ka_BG)-(a_BkG)=Q$.

\section{Massey-Omura}
Neither public key nor symmetric key cryptosystem. Uses $\mathbb{F}_q$. Alice
and Bob each pick an $e$ where $\text{gcd}(e,q-1) = 1$. Each computes their own
$e^{-1} = d$.

\begin{enumerate}
  \item A reduces $m^{e_A}$ in $\mathbb{F}_q$. Sends to B.
  \item B computes $m^{e_Ae_B}$, sends to A.
  \item A computes $m^{e_Ae_Bd_A} = m^{e_B}$. Sends to B.
  \item B computes $m^{e_Bd_b} = m$.
\end{enumerate}

\subsection{ElGamal Signatures}
Have a large prime $p$ and $g$, a generator of $\mathbb{F}_p^*$. These are
secret keys, $a$, and public keys $g^a$. Alice wants to sign a hash of a
message and send it to Bob. Let $S$ be an encoding of the hash as an integer
$1<S<p$. Alice picks a random session key with $1<<k<<p$. Should have
$\text{gcd}(k,p-1)=1$. Alice reduces $g^k$ in $\mathbb{F}_p$ and gets $r$.
Solves $S \equiv a_Ar+kx(\text{mod}p-1)$ for $x$. Alice ends $r$, $x$, $S$ to
Bob as a signature.

Bob computes $(g^{a_A})^rr^x(\text{mod}p)$ and $g^S(\text{mod}p)$.

\section{Hashes and MACs}
A hash function $f(x)$ sends $m+t$-bit strings to $t$-bit strings. When used
in cryptography, should have 3 properties. A hash algorithm $H(x)$ is built up
from a hash function and sends strings of arbitrary length to $t$-bit strings.
Has the same three properties.

\begin{enumerate}
  \item A hash algorithm is said to have the one way property if, given the
    output $y$, it is difficult to find any input $x$ such that $H(x)=y$.
  \item It has the weakly collision free property if, given $x$, it is very
    difficult to find $x' \neq x$ such that $H(x')=H(x)$.
  \item It has the strongly collision free property if it is very difficult to
    find any pair $x \neq x'$ with the same hash.
\end{enumerate}

To create a hash algorithm from a hash function: Use a hash function with two
inputs, an $m$-bit string and a $t$-bit string. Output is a $t$ bit string.
Assume the message, $M$, has more than $m$ bits. Break $M$ into $m$-bit blocks,
padding the last block if necessary.

If we replace the IV with a secret shared key, a hash algorithm is called a
message authentication code (MAC).

\section{Certificates}
An entity can tie a digital transaction to itself with a certificate.
Certification Authorities (CA) use their private keys to sign a certificate
connection an entity with a public key.

Certificates have a hash value of everything above the hash value and then a
signature using that hash value.

\section{SSL}
Two people in the interaction: user and "website". User sends public key to the
website who responds with their own certificate. User checks the certificate
and website generates a block cipher key (AES) and a MAC key. These keys are
encrypted using the user's $e$ and $n$. User thn encrypts a message, hashes
(MAC) it and signs the hash. Website verifies the information.

Know a few things:
\begin{enumerate}
  \item Message wasn't tampered with (integrity).
  \item Message came from the person that owns the $e$ and the $n$
    (authentication).
  \item Repudiation is difficult because of the signature.
\end{enumerate}

\end{document}
