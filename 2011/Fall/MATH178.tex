\documentclass{article}
\usepackage{amsmath,amsthm,amsfonts}
\usepackage{fullpage}
\setlength{\parindent}{0pt}
\setlength{\parskip}{1ex}

\begin{document}
\title{MATH 178 (Cryptography) Study Guide}
\author{Chris Barna (chris@unbrain.net)}
\date{Fall 2011}

\maketitle

\section{Types of Ciphers/Systems}
\begin{enumerate}
  \item \textbf{Stream Ciper}: Operates on a message symbol by symbol or, now,
    bit by bit.
  \item \textbf{Block Cipher}: Operates on blocks of symbols or bits.
  \item \textbf{Transposition Cipher}: Rearranges letters in a plaintext.
  \item \textbf{Substitution Cipher}: Replaces letters, symbols, etc. with
    others but doesn't change the order.
  \item \textbf{Product Cipher}: Alternates substitution and transposition.
  \item \textbf{Symmetric Key Cryptosystem}: Requires a secret shared key. Key
    must be agreed upon ahead of time.
  \item \textbf{Public Key Cryptosystem}: Each user has an encrypting key which
    is published and a decrypting key which is not.
\end{enumerate}

\section{Historical Ciphers}

\subsection{Spartan Scytale}
Example of a transposition cipher. Letters were written on a long strip of
leather that was wrapped around a cylinder. The diameter of the cylinder was
the key to decreption.

\subsection{Playfair}
Key is a word. Write the word, without repeating letters, as the first letters
of a 5x5 square. Write the remaining letters of the alphabet in order to fill
in the square. I and J occupy the same space.

\textbf{To encrypt}: Working two letters at a time, make a box with a pair of
letters and take the letters from the opposite corners of the box. If the
letters are in the same row, replace each letter with the letter to the right.
If the letters are in the same column, replace each letter by the letter below
it.

\subsection{ADFGVX}
Product cipher with fixed table.

\textbf{To encrypt}: Replace a PT letter/digit by the (row, column) pair.
Choose a key with no repeated letters (ex: DEUTSCH). Number the letters in the
key alphabetically. Put partial ciphertext row by row under the key. Write the
columns numerically in increasing order.

\subsection{Shift Enciphering Transformation}
If we have $N$ letters, shift enciphering $C = P + b (\text{mod} N)$ where $b$
is the encrypting key. Decrypting key is $-b$.

\subsection{Affine Cipher}
$C = aP + b (\text{mod} N$ where $a$, $b$ is the encrypting key.
$\text{gcd}(a,n) = 1$ otherwise it doesn't work. Total number of key pairs is
thus $\phi(N) \times (N)$.

\subsection{Digraphs}
Compute digraphs with $C = a(26) + b$.

\subsection{Stream Cipher}
Plaintext is encoded in ASCII and then this is XORed with a binary keystream.

\subsection{Pseudo-Random Bit String Generation}
Say $p$ is a large prime for which $2$ generates $\mathbb{F}_p^*$ and assume
$q=2p+1$ is also prime. Let $g$ be a specified generater of $\mathbb{F}_q^*$.

The key $k$ is a number with $\text{gcd}(k,2p)=1$. Let $s_1=g^k$ in
$\mathbb{F}_q^*$ and $s_1 (\text{mod} 2) = k_1$. For $i \geq 1$ let
$S_{i+1}=S_i^2 \in \mathbb{F}_q^*$.

\section{Number Theory}

\subsection{Modular Arithmetic}
In general, working mod $m$ breaks integers into $m$ subsets. Each subset
contains exactly one representative in the interval $0,...,m-1$.
$\mathbb{Z}/m\mathbb{Z}$ has $m$ elements.

\subsubsection{Solving $ax \equiv b (\text{mod} m)$}
\begin{enumerate}
  \item If $\text{gcd}(a,m) = 1$ then $a^{-1}$ exists so
    $x\equiv a^-1 \cdot b$.
  \item If $\text{gcd}(a,m) = g > 1$ and if $g | b$ then same as
    $\frac{a}{g}x\equiv\frac{b}{g} (\text{mod} \frac{m}{g})$
  \item If $\text{gcd}(a,m) = g > 1$ and if $g \not| b$ then no solution.
\end{enumerate}

\subsubsection{Properties}
\begin{enumerate}
  \item $a \equiv a (\text{mod} m)$.
  \item if $a \equiv b (\text{mod} m)$ then $b \equiv a (\text{mod} m)$.
  \item if $a \equiv b (\text{mod} m)$ and $b \equiv c (\text{mod} m)$, then
    $a \equiv c (\text{mod} m)$.
  \item if $a \equiv b (\text{mod} m)$ and $c \equiv d (\text{mod} m)$, then
    $a \pm c \equiv b \pm d (\text{mod} m)$.
\end{enumerate}

\subsection{$\phi(n)$}
Counts number of coprimes less than $n$.

\begin{enumerate}
  \item $\phi(p) = p-1$
  \item $\phi(p^r) = p^{r-1}(p-1)$
  \item If $\text{gcd}(m,n) = 1$ then $\phi(mn)= \phi(m) \phi(n)$
\end{enumerate}

\subsection{$\mathbb{Z} / n\mathbb{Z}$}
An element $x$ of $\mathbb{Z}/m\mathbb{Z}$ has a multiplicative inverse denoted
$x^{-1}$. in $\mathbb{Z}/m\mathbb{Z}$ if $\text{gcd}(x,m)=1$. The elements of
$\mathbb{Z}/m\mathbb{Z}$ with inverses are denoted $\mathbb{Z}/m\mathbb{Z}^*$.

Operations in $\mathbb{Z}/m\mathbb{Z}$ are $+, -, \cdot$. Operations in
$\mathbb{Z}/m\mathbb{Z}^*$ are $\cdot, \div$.

Size of $\mathbb{Z}/n\mathbb{Z}^*$ is $\phi(n)$.

\subsection{Finite Fields}
If $p$ is prime then $\mathbb{Z}/p\mathbb{Z}$ is denoted $\mathbb{F}_p$. All
non-zero elements, $\alpha$, have $\text{gcd}(\alpha, p) = 1$.

\subsubsection{Finite Fields over Polynomials}
Let $\mathbb{F}_2[x]$ be the set of polynomials with coefficients in
$\mathbb{F}_2$. If you take $\mathbb{F}_2[x]$ and reduce it by an irreducable
polynomial of degree $d$, you get all the polynomials of degree lower than $d$.
This is denoted $\mathbb{F}_{2^d}$. $\mathbb{F}_{2^d}^{*}$ is the non-zero
elements. Easy to represent these polynomials in a computer.

\section{S-AES}
In 1997, NIST held a competition to replace DES. In 2001, NIST chose 128-bit
Rijndael with 128 bit key to become the Advanced Encryption Standard.

AES has full diffusion after two rounds and is non-linear allowing for
reasonable security. AES is also fairly efficient. Each step can be broken into
independent calculations.

Simplified AES (Schaefer, Musa, Wedig) operates over the finite fielt
$\mathbb{F}_16 = \mathbb{F}_2[2]/(x^4+x+1)$.

\subsection{Operations}
Each function operates on a state, consisting of 4 nibbles.

\subsubsection{S-box}
Maps from nibbles to nibbles. Can be inverted. Look it up.

\subsubsection{Key Expansion}
Original key fills $W[0]$ and $W[1]$. For $W[i]$ where $2 \leq i \leq 5$:

\begin{enumerate}
  \item if $i \equiv 0 (\text{mod} 2)$ then
    $W[i]=W[i-2]\oplus\text{RCON}(i/2)\oplus\text{SubNib}(\text{RotNib}(W[i-1]))$
  \item if $i \not\equiv 0 (\text{mod} 2)$ then $W[i]=W[i-2]\oplus W[i-1]$
\end{enumerate}

$\text{RotNib}(N_0N_1) = N_1N_0$.
$\text{SubNib}(N_0N_1) = \text{S-box}(N_0)\text{S-box}(N_1)$.
$\text{RCON}[1]=10000000$, $\text{RCON}[2]=00110000$.

\subsubsection{$A_{K_i}$}
Add key. XOR $K_i$ with the state.

\subsubsection{$NS$}
Nibble substitution. Replace each nibble $N_i$ in a state by
$\text{S-box}(N_i)$. Do not change the order of the nibbles.

\subsubsection{$SR$}
Shift row. Shift the bottom row of the state.

\subsubsection{$MC$}
Mix column. Look it up.


\end{document}
